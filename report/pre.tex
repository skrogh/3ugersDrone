\documentclass[a4paper]{article}
\usepackage[utf8]{inputenc} % Skal passe til editorens indstillinger
\usepackage[english]{babel} % danske overskrifter


\newcommand{\name}{Søren Krogh Andersen \& Carsten Lau Nielsen}
\newcommand{\stnumber}{s123369, s123161}
\newcommand{\course}{Special Course}
\newcommand{\university}{Danmarks Tekniske Universitet}
\newcommand{\studyline}{DTU Elektroteknologi}
\newcommand{\assignment}{Attitude controller for multirotor}
\renewcommand{\date}{\today} %If another date, than that of today is desiered


% Palatino for rm and math | Helvetica for ss | Courier for tt
\usepackage{mathpazo} % math & rm
\linespread{1.05}        % Palatino needs more leading (space between lines)
\usepackage{palatino} % tt
\normalfont
\usepackage[T1]{fontenc}

\usepackage{graphicx}%allerese hentet % indsættelse af billeder
\usepackage{epstopdf} %Tilfj "--enable-write18" i argumentet for LaTex build. Dette vil konvertere .eps figurer til pdf-format
\graphicspath{{./picture/}} % stivej til bibliotek med figurer
\usepackage{subcaption} %Til gruppering af figurer
\usepackage{amsmath} %matpakke
\usepackage{amsfonts} %
\usepackage{amssymb} %
\usepackage{steinmetz} % flere matematik symboler
\usepackage{polynom} %for displaying polynom division
\usepackage{mathtools} % matematik - understøtter muligheden for at bruge \eqref{}
\usepackage{float}
\usepackage{placeins}
\usepackage{hhline}

%
\usepackage[usenames,dvipsnames]{xcolor}
\usepackage[compact,explicit]{titlesec}% http://ctan.org/pkg/titlesec
%

%---------%
%Easy edit%
%---------%

%Section formating. arg1 is supplied when making section
\newcommand\presectionnumber[1]{~~}
\newcommand\postsectionnumber[1]{}
\newcommand\midlesection[1]{#1}
\newcommand\sectionnum[1]{\arabic{#1}}
\newcommand\subsectionnum[1]{\arabic{#1}}
\newcommand\subsubsectionnum[1]{\alph{#1}}



%------------%
%setion setup%
%------------%
\renewcommand\thesection{Opgave~\sectionnum{section}} %pas p�, kun i matematik
\renewcommand\thesubsection{\thesection,~\subsectionnum{subsection}}
\definecolor{MagRed}{RGB}{190,40,15}
\definecolor{MathGreen}{RGB}{82,164,0}

\titleformat{\section}{\normalfont\sffamily\large\bfseries\color{MathGreen}}{}{0pt}{|\kern-0.15ex|\kern-0.15ex|\kern-0.15ex|\presectionnumber{#1}\sectionnum{section}\postsectionnumber{#1}\qquad\quad\midlesection{#1}\label{sec:\sectionnum{section}}}
\titleformat{\subsection}[runin]{\large\bfseries}{}{10pt}{\sectionnum{section}.\subsectionnum{subsection})~#1\label{sec:\sectionnum{section}.\subsectionnum{subsection}}}
\titleformat{\subsubsection}[runin]{\itshape}{}{0pt}{\subsectionnum{subsection},\subsubsectionnum{subsection}~#1\label{sec:\sectionnum{section}.\subsectionnum{subsection}.\subsubsectionnum{subsubsection}}}
%\titleformat{\subsubsection}{\bfseries}{}{0pt}{\alph{subsection}.\arabic{subsubsection})\qquad\quad#1\label{\arabic{section}\alph{subsection}\arabic{subsubsection}}}

%----------%
%page setup%
%----------%
\textwidth = 400pt
\marginparwidth = 86pt
\hoffset = -25pt
\voffset= -30pt
\textheight = 670pt

%--------%
%hyperref%
%--------%
\newcommand{\HRule}{\rule{\linewidth}{0.5mm}}
\usepackage{fancyhdr}
\usepackage[plainpages=false,pdfpagelabels,pageanchor=false]{hyperref} % aktive links
\hypersetup{%
  pdfauthor={\name - \stnumber},
  pdftitle={\assignment},
  pdfsubject={\course} }
%\usepackage{memhfixc}% rettelser til hyperref

%-------------%
%Headder setup%
%-------------%
\fancyhf{} % tom header/footer
\fancyhfoffset{20pt}
\fancyhfoffset{20pt}
\fancyhead[OL]{\name \\ \course}
\fancyhead[OC]{Date \\ \date}
\fancyhead[OR]{\university\\ \studyline}
\fancyfoot[FL]{}
\fancyfoot[FC]{\thepage}
\fancyfoot[FR]{}
\renewcommand{\headrulewidth}{0.4pt}
\renewcommand{\footrulewidth}{0.4pt}
\pagestyle{fancy}
 % style setup

%Listings%
\usepackage{listingsutf8}
\usepackage[framed,numbered]{matlab-prettifier}


%setup listings
\lstset{language=Matlab,
  extendedchars=true,
  language=Octave,                % the language of the code
  basicstyle=\ttfamily\footnotesize,           % the size of the fonts that are
  % used for the code
  numbers=left,                   % where to put the line-numbers
  numberstyle=\tiny\color{gray},  % the style that is used for the line-numbers
  stepnumber=2,                   % the step between two line-numbers. If it's 1, each line 
                                  % will be numbered
  numbersep=5pt,                  % how far the line-numbers are from the code
  backgroundcolor=\color{white},      % choose the background color. You must add \usepackage{color}
  showspaces=false,               % show spaces adding particular underscores
  showstringspaces=false,         % underline spaces within strings
  showtabs=false,                 % show tabs within strings adding particular underscores
  frame=single,                   % adds a frame around the code
  rulecolor=\color{black},        % if not set, the frame-color may be changed on line-breaks within not-black text (e.g. comments (green here))
  tabsize=4,                      % sets default tabsize to 2 spaces
  captionpos=b,                   % sets the caption-position to bottom
  breaklines=true,                % sets automatic line breaking
  breakatwhitespace=false,        % sets if automatic breaks should only happen at whitespace
  title=\lstname,                   % show the filename of files included with \lstinputlisting;
                                  % also try caption instead of title
  %keywordstyle=\color{blue},          % keyword style
  %commentstyle=\color{dkgreen},       % comment style
  %stringstyle=\color{mauve},         % string literal style
  escapeinside={\%*}{*)},            % if you want to add LaTeX within your code
  morekeywords={*,...},              % if you want to add more keywords to the set
  deletekeywords={...}              % if you want to delete keywords from the given language
}
\lstset{literate=
  {á}{{\'a}}1 {é}{{\'e}}1 {í}{{\'i}}1 {ó}{{\'o}}1 {ú}{{\'u}}1
  {Á}{{\'A}}1 {É}{{\'E}}1 {Í}{{\'I}}1 {Ó}{{\'O}}1 {Ú}{{\'U}}1
  {à}{{\`a}}1 {è}{{\`e}}1 {ì}{{\`i}}1 {ò}{{\`o}}1 {ù}{{\`u}}1
  {À}{{\`A}}1 {È}{{\'E}}1 {Ì}{{\`I}}1 {Ò}{{\`O}}1 {Ù}{{\`U}}1
  {ä}{{\"a}}1 {ë}{{\"e}}1 {ï}{{\"i}}1 {ö}{{\"o}}1 {ü}{{\"u}}1
  {Ä}{{\"A}}1 {Ë}{{\"E}}1 {Ï}{{\"I}}1 {Ö}{{\"O}}1 {Ü}{{\"U}}1
  {â}{{\^a}}1 {ê}{{\^e}}1 {î}{{\^i}}1 {ô}{{\^o}}1 {û}{{\^u}}1
  {Â}{{\^A}}1 {Ê}{{\^E}}1 {Î}{{\^I}}1 {Ô}{{\^O}}1 {Û}{{\^U}}1
  {œ}{{\oe}}1 {Œ}{{\OE}}1 {æ}{{\ae}}1 {Æ}{{\AE}}1 {ß}{{\ss}}1
  {ç}{{\c c}}1 {Ç}{{\c C}}1 {ø}{{\o}}1 {å}{{\r a}}1 {Å}{{\r A}}1
  {€}{{\EUR}}1 {£}{{\pounds}}1
}

 \lstloadlanguages{% Check Dokumentation for further languages ...
         %[Visual]Basic
         %Pascal
         %C
         %C++
         %XML
         %HTML
         %Java
         %VHDL
         Matlab
 }
 %Listings slut%









%Matematik hurtige ting
%fed
\renewcommand\vec[1]{\boldsymbol{#1}}
\newcommand\matr[3]{{}_{#2}\mathbf{#1}{}_{#3}}
\newcommand\facit[1]{\underline{\underline{#1}}}
%\renewcommand\d[3]{\frac{\mbox{d}^{#3}#1(#2)}{\mbox{d}#2^{#3}}}
%underline
%\renewcommand\vec[1]{\underline{#1}}
%\newcommand\matr[3]{{}_{#2}\underline{\underline{#1}}{}_{#3}}

\renewcommand\matrix[4]{ %{alignment}{to space}{from space}{matrix}
{\vphantom{\left[\begin{array}{#1}#4\end{array}\right]}}_{#2}\kern-0.5ex
\left[\begin{array}{#1}
#4
\end{array}\right]_{#3}
}
\newcommand\e[0]{\mbox{e}}
\newcommand\E[1]{\cdot 10^{#1}}
\newcommand\im[0]{i}

\newcommand\Jaco{\mbox{Jacobi}}
\newcommand\del[2]{\frac{\partial {#1}}{\partial {#2}}}
\newcommand\abs[1]{\left| {#1} \right|}
\newcommand\stdfig[4]{ %width,img,cap,lab
\begin{figure}[H]
\centering
\includegraphics[width={#1}\textwidth]{#2}
\caption{#3}
\label{#4}
\end{figure}
}
\newcommand\stdfignoscale[3]{ %img,cap,lab
\begin{figure}[H]
\centering
\includegraphics{#1}
\caption{#2}
\label{#3}
\end{figure}
}
\newcommand\diff{\dot}
\newcommand\ddiff{\ddot}
\newcommand\dddiff{\dddot}
\newcommand\ddddiff{\ddddot}






% How to make ref to books or urls in bib
%\citetitle[fx: page 1]{name of ref in bib}
