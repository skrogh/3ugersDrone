\documentclass[a4paper]{article}
\usepackage[utf8]{inputenc} % Skal passe til editorens indstillinger
\usepackage[english]{babel} % danske overskrifter


\newcommand{\name}{Søren Krogh Andersen \& Carsten Lau Nielsen}
\newcommand{\stnumber}{s123369, s123161}
\newcommand{\course}{Control of Agile Quadrotor}
\newcommand{\university}{Technical University of Denmark}
\newcommand{\studyline}{Department of Automation and Control}
\newcommand{\assignment}{Attitude controller for multirotor}
\renewcommand{\date}{\today} %If another date, than that of today is desiered


% Palatino for rm and math | Helvetica for ss | Courier for tt
\usepackage{mathpazo} % math & rm
\linespread{1.05}        % Palatino needs more leading (space between lines)
\usepackage{palatino} % tt
\normalfont
\usepackage[T1]{fontenc}

\usepackage{graphicx}%allerese hentet % indsættelse af billeder
\usepackage{epstopdf} %Tilfj "--enable-write18" i argumentet for LaTex build. Dette vil konvertere .eps figurer til pdf-format
\graphicspath{{./picture/}} % stivej til bibliotek med figurer
\usepackage{subcaption} %Til gruppering af figurer
\usepackage{amsmath} %matpakke
\usepackage{amsfonts} %
\usepackage{amssymb} %
\usepackage{steinmetz} % flere matematik symboler
\usepackage{polynom} %for displaying polynom division
\usepackage{mathtools} % matematik - understøtter muligheden for at bruge \eqref{}
\usepackage{float}
\usepackage{placeins}
\usepackage{hhline}

%
\usepackage[usenames,dvipsnames]{xcolor}
\usepackage[compact,explicit]{titlesec}% http://ctan.org/pkg/titlesec
%

%---------%
%Easy edit%
%---------%

%Section formating. arg1 is supplied when making section
\newcommand\presectionnumber[1]{~~}
\newcommand\postsectionnumber[1]{}
\newcommand\midlesection[1]{#1}
\newcommand\sectionnum[1]{\arabic{#1}}
\newcommand\subsectionnum[1]{\arabic{#1}}
\newcommand\subsubsectionnum[1]{\alph{#1}}



%------------%
%setion setup%
%------------%
\renewcommand\thesection{Opgave~\sectionnum{section}} %pas p�, kun i matematik
\renewcommand\thesubsection{\thesection,~\subsectionnum{subsection}}
\definecolor{MagRed}{RGB}{190,40,15}
\definecolor{MathGreen}{RGB}{82,164,0}

\titleformat{\section}{\normalfont\sffamily\large\bfseries\color{MathGreen}}{}{0pt}{|\kern-0.15ex|\kern-0.15ex|\kern-0.15ex|\presectionnumber{#1}\sectionnum{section}\postsectionnumber{#1}\qquad\quad\midlesection{#1}\label{sec:\sectionnum{section}}}
\titleformat{\subsection}[runin]{\large\bfseries}{}{10pt}{\sectionnum{section}.\subsectionnum{subsection})~#1\label{sec:\sectionnum{section}.\subsectionnum{subsection}}}
\titleformat{\subsubsection}[runin]{\itshape}{}{0pt}{\subsectionnum{subsection},\subsubsectionnum{subsection}~#1\label{sec:\sectionnum{section}.\subsectionnum{subsection}.\subsubsectionnum{subsubsection}}}
%\titleformat{\subsubsection}{\bfseries}{}{0pt}{\alph{subsection}.\arabic{subsubsection})\qquad\quad#1\label{\arabic{section}\alph{subsection}\arabic{subsubsection}}}

%----------%
%page setup%
%----------%
\textwidth = 400pt
\marginparwidth = 86pt
\hoffset = -25pt
\voffset= -30pt
\textheight = 670pt

%--------%
%hyperref%
%--------%
\newcommand{\HRule}{\rule{\linewidth}{0.5mm}}
\usepackage{fancyhdr}
\usepackage[plainpages=false,pdfpagelabels,pageanchor=false]{hyperref} % aktive links
\hypersetup{%
  pdfauthor={\name - \stnumber},
  pdftitle={\assignment},
  pdfsubject={\course} }
%\usepackage{memhfixc}% rettelser til hyperref

%-------------%
%Headder setup%
%-------------%
\fancyhf{} % tom header/footer
\fancyhfoffset{20pt}
\fancyhfoffset{20pt}
\fancyhead[OL]{\name \\ \course}
\fancyhead[OC]{Date \\ \date}
\fancyhead[OR]{\university\\ \studyline}
\fancyfoot[FL]{}
\fancyfoot[FC]{\thepage}
\fancyfoot[FR]{}
\renewcommand{\headrulewidth}{0.4pt}
\renewcommand{\footrulewidth}{0.4pt}
\pagestyle{fancy}
 % style setup

%Listings%
\usepackage{listingsutf8}
\usepackage[framed,numbered]{matlab-prettifier}


%setup listings
\lstset{language=Matlab,
  extendedchars=true,
  language=Octave,                % the language of the code
  basicstyle=\ttfamily\footnotesize,           % the size of the fonts that are
  % used for the code
  numbers=left,                   % where to put the line-numbers
  numberstyle=\tiny\color{gray},  % the style that is used for the line-numbers
  stepnumber=2,                   % the step between two line-numbers. If it's 1, each line 
                                  % will be numbered
  numbersep=5pt,                  % how far the line-numbers are from the code
  backgroundcolor=\color{white},      % choose the background color. You must add \usepackage{color}
  showspaces=false,               % show spaces adding particular underscores
  showstringspaces=false,         % underline spaces within strings
  showtabs=false,                 % show tabs within strings adding particular underscores
  frame=single,                   % adds a frame around the code
  rulecolor=\color{black},        % if not set, the frame-color may be changed on line-breaks within not-black text (e.g. comments (green here))
  tabsize=4,                      % sets default tabsize to 2 spaces
  captionpos=b,                   % sets the caption-position to bottom
  breaklines=true,                % sets automatic line breaking
  breakatwhitespace=false,        % sets if automatic breaks should only happen at whitespace
  title=\lstname,                   % show the filename of files included with \lstinputlisting;
                                  % also try caption instead of title
  %keywordstyle=\color{blue},          % keyword style
  %commentstyle=\color{dkgreen},       % comment style
  %stringstyle=\color{mauve},         % string literal style
  escapeinside={\%*}{*)},            % if you want to add LaTeX within your code
  morekeywords={*,...},              % if you want to add more keywords to the set
  deletekeywords={...}              % if you want to delete keywords from the given language
}
\lstset{literate=
  {á}{{\'a}}1 {é}{{\'e}}1 {í}{{\'i}}1 {ó}{{\'o}}1 {ú}{{\'u}}1
  {Á}{{\'A}}1 {É}{{\'E}}1 {Í}{{\'I}}1 {Ó}{{\'O}}1 {Ú}{{\'U}}1
  {à}{{\`a}}1 {è}{{\`e}}1 {ì}{{\`i}}1 {ò}{{\`o}}1 {ù}{{\`u}}1
  {À}{{\`A}}1 {È}{{\'E}}1 {Ì}{{\`I}}1 {Ò}{{\`O}}1 {Ù}{{\`U}}1
  {ä}{{\"a}}1 {ë}{{\"e}}1 {ï}{{\"i}}1 {ö}{{\"o}}1 {ü}{{\"u}}1
  {Ä}{{\"A}}1 {Ë}{{\"E}}1 {Ï}{{\"I}}1 {Ö}{{\"O}}1 {Ü}{{\"U}}1
  {â}{{\^a}}1 {ê}{{\^e}}1 {î}{{\^i}}1 {ô}{{\^o}}1 {û}{{\^u}}1
  {Â}{{\^A}}1 {Ê}{{\^E}}1 {Î}{{\^I}}1 {Ô}{{\^O}}1 {Û}{{\^U}}1
  {œ}{{\oe}}1 {Œ}{{\OE}}1 {æ}{{\ae}}1 {Æ}{{\AE}}1 {ß}{{\ss}}1
  {ç}{{\c c}}1 {Ç}{{\c C}}1 {ø}{{\o}}1 {å}{{\r a}}1 {Å}{{\r A}}1
  {€}{{\EUR}}1 {£}{{\pounds}}1
}

 \lstloadlanguages{% Check Dokumentation for further languages ...
         %[Visual]Basic
         %Pascal
         %C
         %C++
         %XML
         %HTML
         %Java
         %VHDL
         Matlab
 }
 %Listings slut%









%Matematik hurtige ting
%fed
\renewcommand\vec[1]{\mathbf{#1}}
\newcommand\matr[3]{{}_{#2}\mathbf{#1}{}_{#3}}
\newcommand\facit[1]{\underline{\underline{#1}}}
%\renewcommand\d[3]{\frac{\mbox{d}^{#3}#1(#2)}{\mbox{d}#2^{#3}}}
%underline
%\renewcommand\vec[1]{\underline{#1}}
%\newcommand\matr[3]{{}_{#2}\underline{\underline{#1}}{}_{#3}}

\renewcommand\matrix[4]{ %{alignment}{to space}{from space}{matrix}
{\vphantom{\left[\begin{array}{#1}#4\end{array}\right]}}_{#2}\kern-0.5ex
\left[\begin{array}{#1}
#4
\end{array}\right]_{#3}
}
\newcommand\e[0]{\mbox{e}}
\newcommand\E[1]{\cdot 10^{#1}}
\newcommand\im[0]{i}

\newcommand\Jaco{\mbox{Jacobi}}
\newcommand\del[2]{\frac{\partial {#1}}{\partial {#2}}}
\newcommand\abs[1]{\left| {#1} \right|}
\newcommand\stdfig[4]{ %width,img,cap,lab
\begin{figure}[H]
\centering
\includegraphics[width={#1}\textwidth]{#2}
\caption{#3}
\label{#4}
\end{figure}
}
\newcommand\stdfignoscale[3]{ %img,cap,lab
\begin{figure}[H]
\centering
\includegraphics{#1}
\caption{#2}
\label{#3}
\end{figure}
}
\newcommand\diff{\dot}
\newcommand\ddiff{\ddot}
\newcommand\dddiff{\dddot}
\newcommand\ddddiff{\ddddot}






% How to make ref to books or urls in bib
%\citetitle[fx: page 1]{name of ref in bib}

\begin{document}
\begin{titlepage}
\centering \parindent=0pt

\vspace*{\stretch{1}} \HRule\\[1cm]\Huge
\course\\[0.7cm]
\large \assignment\\[1cm]
\HRule\\[4cm]  
%\includegraphics[width=6cm]{picture}\\ Use this if you want a picture on the frontpage
\name\\
\stnumber

\vspace*{\stretch{2}} \normalsize %
\begin{flushleft}

\date \end{flushleft}
\vspace*{\stretch{2}} \normalsize
\begin{flushright}
\includegraphics[width=6cm]{./dtu.eps}\\
\end{flushright}
\end{titlepage}

\newpage

%Quick hardware walktrough
\section{ Introduction }



\section{Hardware description}
Most of the hardware was contructed before the start of the course, mainly
because we needed a custom PCB fabricated and lead times for PCBs from China are
unpredictable and can be long.
Likewise a lot of 3D printing was needed for the frame, and this takes quite
some time (total printing time for the parts used was about 14h, then comes
part design, prototypes and printer setup).

\subsection{ Flight controller }
The main processor of the fligt controller is a \emph{Texas
Instruments} \emph{TM4C123GH6 tiva series ARM CORTEX M4F} processor.
This processor was chosen due to ease of use.
The chip is avaliable from most distributors and cheap development boards,
\emph{TI Launchpad}s, can be used as debuggers. Further more a huge driver
library (\emph{TivaWare}) for hardware abstraction is available from \emph{TI}s
webpage.

The flighcontroller has a varity of IO ports: PWM outputs, I2C, CAN, UART,
analog and digital inpus, a special port for conneting to the \emph{Hardkernel
oDroid U3} Linux computer via SPI, SPI connection for a radio module, and
onboard IMU (\emph{Invensense mpu9250}) and barometer (\emph{Freescale
MPL3115A2}).

A close-up of the populated fligtcontroller PCB can be seen in figure
\ref{fig:flightcontroller}. Note that not all IO modules are populated with the
needed parts. Some mistakes were made in the layout, and thus had to be fixed
(pink wire nest).

\subsection{ Radio }
We chose the \emph{TI CC1101} $433MHz$ radio module, as this was relativly easy
to get working.

For non-tethered telemetry a remote controller was contructed, capable of
sending low amounts of data between the flightcontroller and remote. The remote
has a USB port for dumping telemetry data to a PC and joystics for sending
steering commands to the flight controller.

A picture of the crude radio remote controller can be seen in figure
\ref{fig:remote}

\subsection{ Motors, ESCs, propellers }
Motors and motor drivers were bought from china. The \emph{EMAX MT1806} motors
and reccomended drivers for those were chosen due to good reviews from
hobbyists. Likewise 5030 carbonfiber propellers were chosen, as they are the
reccomended propellers for motors of this size.

\section{ System identification } %change title?
While the differential equations governing attitude are highly nonlinear,
they can be linearized about a working-point, where the angular speed is zero.
Thus the gyroscopic effects can be omitted.

The thrust of each motor is assumed to be linear with the control output and the
retardation can be modelled as a first order lowpass filter.

The differential equations for rotation arround the $x$ and $y$ axes
becomes:
\begin{equation}
 \dddiff\theta_x = \frac{1}{\tau_{motor}} \cdot \left(
 (u_1 + u_2 - u_3 - u_4) G_{thrust} l_{y} \frac{ 1 }{ I_{x} }  - \ddiff\theta_x
 \right) 
\end{equation}
\begin{equation}
 \dddiff\theta_y = \frac{1}{\tau_{motor}} \cdot \left(
 (-u_1 + u_2 + u_3 - u_4) G_{thrust} l_{x} \frac{ 1 }{ I_{y} }  - \ddiff\theta_y
 \right) 
\end{equation}
Where $G_{thrust}$ is the thrust in $N$ produced by a single propeller at full
thrust, $l_{x}$ is the length from the center of mass to the motors along the
$x$-axis, $I_{y}$ is the moment of inertia arround the $y$-axis,
$\tau_{motor}$ is the time constant of the motor and $u_{[1;4]}$ are motor
commands for motor $[1;4]$ in the range $[0;1]$.

Local coordinate system for the drone, and motor numbering is chosen as shown in
figure \ref{fig:axis}
\stdfig{0.8}{axis}{Choise of local coordinate system and motor numbering for
drone}{fig:axis}

If we control the motors in tandem so:
\begin{align*}
u_1 = & u_x - u_y + T\\
u_2 = & u_x + u_y + T\\
u_3 = & -u_x + u_y + T\\
u_4 = & -u_x - u_y + T\\
\end{align*}
Where T is a general thrust command the equatinos simplify to:
\begin{equation}
 \dddiff\theta_x = \frac{1}{\tau_{motor}} \cdot \left(
 4 u_x G_{thrust} l_{y} \frac{ 1 }{ I_{x} }  - \ddiff\theta_x
 \right) 
\end{equation}
\begin{equation}
 \dddiff\theta_y = \frac{1}{\tau_{motor}} \cdot \left(
4 u_y G_{thrust} l_{x} \frac{ 1 }{ I_{y} }  - \ddiff\theta_y
 \right) 
\end{equation}


\section{ Sensor filtering } %change title?
Since the rotating propellers introduce a lot of vibration some sort of
filtering is required. Clasically mechanical filters (foam and rubber-rings) has
been used as a solutuion to this problem. We propose using a high-samplerate
IMU and using a digital filter to supress vibration above the output rate of the
controller. Assuming that the plant dynamics are temporarily linear, we can
perform further state estimation on a downsampled version of the IMU- readings,
allowing for more complex estimators.

As an attitude estimator we use the popular \emph{Madgwic sensor fusion
algorithm}.

\section{ Controller } %change title?

\section{ Results } %change title?



%Novelties:
%*High sample-rate of acc/gyro, downsampled to remove aliasing.
% This results in propper attitude estimation, while allowing for high amounts
% of vibration.
%
% Mechanical dapening still needed to avoid sensor saturation.
%
% - plot of gyro before and after filter
% - plot of gyro with and widout earbuds

%*Hiding dynamics of rotor-spinup, resulting in easy controller design.
% - Copter simulink model
% - Bode plot w/wo controller (inner loop)

%*Controller designed by poleplacement, with state limitations.
% - Copter simulink model
% - Bode plot w/wo controller

%Camerastuff?


\end{document}
